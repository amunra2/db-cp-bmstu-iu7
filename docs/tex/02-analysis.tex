\chapter{Аналитическая часть}

В данном разделе будет проведен анализ существующих решений на русском и зарубежном рынках. Также будет произведена формализация задачи и данных, описание типов пользователей, а также обзор существующих типов баз данных.

\section{Анализ существующих решений}

Существует большое количество сайтов, которые предоставляют возможность найти сервера для той или иной онлайн игры.


\subsection{Российский рынок}

Одним из сайтов для поиска серверов для игры <<Minecraft>> \cite{minecraft} является \newline <<MinecaftRating>> \cite{mine-rating}. При этом интерфейс главной страницы представлен на рисунке \ref{img:mine_rus}. На нем предоставляется большое количество функциональностей, таких как: 

\begin{itemize}
    \item поиск серверов (с сортировкой по самым популярным, по версии, по количеству игроков);
    \item получение полной информации о каждом сервере;
    \item добавление нового сервера;
    \item для зарегестрированных пользователей --- добавление в избранное.
\end{itemize}

При этом не имеется возможности посмотреть сервера для разных платформ.

Для компьютерной игры <<Counter-Strike>> \cite{cs} также существуют сайты с серверами. 
Примером такого сайта является <<Сервера КС>> \cite{servers-cs}. На рисунке \ref{img:cs_rus} представлен интерфейс главной страницы. Для данной игры рынок развит слабее, поэтому функциональностей куда меньше. При этом предоставляются следующие возможности:

\begin{itemize}
    \item просмотр серверов (присутсвует лишь список, поиск и сортировка невозможны); 
    \item для зарегестрированных пользователей --- добавление нового сервера, добавление в избранное.
\end{itemize}

Из-за отсутсвия сортировок серверов, имеется возможность купить место в верху таблицы, чтобы пользователи замечали сначала проплаченные сервера. 

\imgs{mine_rus}{h!}{0.3}{Сайт для поиска серверов для игры <<Minecraft>> на русском рынке}
\imgs{cs_rus}{h!}{0.3}{Сайт для поиска серверов для игры <<Counter-Strike>> на русском рынке}
% ~\\~\\~\\~\\~\\~\\~\\~\\~\\

\subsection{Зарубежный рынок}

Примером сайта для поиска серверов для игры <<Minecraft>> \cite{minecraft} является <<Minecraft Servers>> \cite{mine-servs}. На рисунке \ref{img:mine_eng} представлен интерфейс главной страницы.Зарубежный аналог обладает тем же самым функционалом, что и пример сайта с российского рынка. При этом поиск серверов для различных платформ также отсутсвует.

Для игры <<Counter-Strike>> \cite{cs} зарубежный рынок развит сильнее. Так, примером является сайт <<Game Tracker>> \cite{game-tracker}. Интерфейс главной страницы представлен на рисунке \ref{img:cs_eng}. Он обладает тем же самым функционалом, что и его российский аналог, а также:

\begin{itemize}
    \item присутсвует сортировка по различным параметрам;
    \item существует возможность выбора серверов для различных стран мира.
\end{itemize}

При этом возможность покупки приоритетного места в списке отсутсвует. 

\imgs{mine_eng}{h!}{0.3}{Сайт для поиска серверов для игры <<Minecraft>> на зарубежном рынке}
\imgs{cs_eng}{h!}{0.3}{Сайт для поиска серверов для игры <<Counter-Strike>> на зарубежном рынке}


\subsection*{Вывод}

\textbf{В РАЗРАБОТКЕ:} СОЗДАТЬ ТАБЛИЦУ +/-.

Таким образом, российский и зарубженый рынки предоставляют множество различных сайтов для нахождения серверов для игр. Стоит отметить то, что при этом сайты обладают рядом серьезных проблем.

\begin{enumerate}
    \item Отсутсвует возможность просмотреть списки серверов для разных платформ одной и той же игры.
    \item Недостаточная функциональность для поиска необходимых серверов.
    \item Слабая развитость рынка для некоторых игр.
\end{enumerate}


\section{Формализация задачи}

Должно быть разработано веб-приложение для поиска серверов для игры. При этом ПО должно содержать в себе следующие возможности:

\begin{itemize}
    \item просмотр списка серверов с сортировкой по параметрам;
    \item авторизация и регистрация на сайте;
    \item просмотра списка избранных серверов;
    \item изменение рейтинга серверов;
    \item просмотр списка серверов для определенной платформы;
    \item добавление/изменение/удаление серверов администратором;
    \item изменение ролей пользователей администратором.
\end{itemize}

\subsection{Список серверов}

Основным процессом является просмотр списка серверов. Список серверов представляет из себя таблицу из всех имеющихся серверов в базе данных. Информационные поля сервера указаны в колонках таблицы, а каждый отдельный сервер --- строка этой таблицы.

Также таблица должна иметь возможонсть сортировки по полям сервера (названию, IP-адресу, версии игры, рейтингу сервера) и имени платформы, на которой запущен сервер.


\subsection{Авторизация и регистрация}

Должна быть введена возможность регистрации пользователя на сайте, чтобы открыть ему дополнительные возможности -- просмотр информации о хостинге сервера, списка игроков сервера и добавления сервера в список избранного.


\subsection{Список избранных серверов}

Каждый зарегестрированный пользователь должен обладать возможность добавить интересующий его сервер в список избранных. Это необходимо для того, чтобы пользователь не потерял интересующий его сервер и всегда имел быстрый доступ к информации о нем.


\subsection{Рейтинг серверов}

Рейтинг сервера формируется из количества добавлений данного сервера в список избранных серверов отдельно взятого пользователя. Данный рейтинг выводится в качестве поля в таблице серверов, предоставляя возможность пользователям узнать наиболее популярный сервер. При этом, если пользователь сайта удалил данный сервер из своего списка избранных серверов, то рейтинг сервера понизится.


\subsection{Деление серверов по платформам}

Каждый сервер может находится лишь на одной единственной платформе. Поэтому важно разделить список серверов на отдельные списки для каждой платформы, чтобы пользователь мог выбрать именно те сервера, которые подходят для его рабочего устройства.


\subsection{Добавление/изменение/удаление серверов}

Данной возможностью наделен лишь администратор сайта. Должен быть предоставлен интерфейс для данного процесса. При добавлении/изменении должны быть добавлены ограничения на ввод информации, чтобы предотвратить ошибки ввода, а также недопустить появления серверов с таким же названием или на том же IP-адресе. При удалении должно быть реализовано подтверждение удаления сервера, чтобы предотвратить случайные нажатия.


\subsection{Изменение ролей зарегестрированных пользователей}

Должна быть введена возможность изменения роли зарегестрированного пользователя. При этом администратор не может изменить собственную роль, а также администратору с никнеймом <<admin>> должно быть запрещено изменять роль, чтобы на сайте был всегда, как минимум, один администратор.


\section{Описание типов пользовтелей}

\textbf{В РАЗРАБОТКЕ:} СОЗДАТЬ ТАБЛИЦУ.

В данной задаче должно быть выделено 3 типа пользователей (таблица \ref{tbl:users_types}). 

% \begin{center}
%     \captionsetup{justification=raggedright,singlelinecheck=off}
%     \begin{longtable}[c]{|p{5cm}|p{11cm}|}
%     \caption{Типы пользователей\label{tbl:users_types}}\\ \hline
%         \textbf{Тип} & \textbf{Функциональность} \\ \hline

%         Гость \par (неавторизованный пользователь) &
%             \begin{minipage}[t]{\linewidth}
%                 \begin{itemize}[nosep,after=\strut]
%                     \item Просмотр списка серверов
%                     \item Выбор платформы для списка серверов
%                     \item Сортировка списка серверов по доступным критериям
%                     \item Регистрация
%                     \item Авторизация
%                 \end{itemize}
%             \end{minipage} 
%         \\ \hline

%         Авторизованный \par пользователь &
%             \begin{minipage}[t]{\linewidth}
%                 \begin{itemize}[nosep,after=\strut]
%                     \item Добавление сервера в список избранных серверов
%                     \item Выбор платформы для списка серверов
%                     \item Просмотр списка серверов
%                     \item Сортировка списка серверов по доступным критериям
%                     \item Просмотр информации о хостинге сервера, а также списка игроков сервера
%                 \end{itemize}
%             \end{minipage} 
%         \\ \hline

%         Администратор &
%             \begin{minipage}[t]{\linewidth}
%                 \begin{itemize}[nosep,after=\strut]
%                     \item Добавление нового сервера на сайт
%                     \item Удаление сервера с сайта
%                     \item Изменение информации о сервере
%                     \item Изменение ролей зарегестрированных пользователей
%                     \item Добавление сервера в список избранных серверов
%                     \item Просмотр списка серверов
%                     \item Выбор платформы для списка серверов
%                     \item Сортировка списка серверов по доступным критериям
%                     \item Просмотр информации о хостинге сервера, а также списка игроков сервера
%                 \end{itemize}
%             \end{minipage} 
%         \\ \hline
% \end{longtable}
% \end{center}

На рисунках \ref{img:usecase_guest}-\ref{img:usecase_admin} представлены Use-Case диаграммы выделенных типов пользователей.

\imgs{usecase_guest}{h!}{0.4}{Use-Case диаграмма для неавторизованного пользователя}
\imgs{usecase_user}{h!}{0.4}{Use-Case диаграмма для авторизованного пользователя}
\imgs{usecase_admin}{h!}{0.4}{Use-Case диаграмма для администратора}


\section{Формализация данных}

\textbf{В РАЗРАБОТКЕ:} СОЗДАТЬ ТАБЛИЦУ.

В соответствии с задачей и типами пользователей, база данных должна содержать следующие модели (таблица \ref{tbl:db_tables})

% \begin{center}
%     \captionsetup{justification=raggedright,singlelinecheck=off}
%     \begin{longtable}[c]{|p{5cm}|p{11cm}|}
%     \caption{Типы пользователей\label{tbl:db_tables}}\\ \hline
%         \textbf{Таблица} & \textbf{Данные} \\ \hline

%         Серевер &
%             \begin{minipage}[t]{\linewidth}
%                 \begin{itemize}[nosep,after=\strut]
%                     \item ID
%                     \item Название
%                     \item IP-адрес
%                     \item Версия игры
%                     \item Рейтинг
%                     \item ID платформы
%                     \item ID хостинга
%                 \end{itemize}
%             \end{minipage}
%         \\ \hline

%         Платформа &
%             \begin{minipage}[t]{\linewidth}
%                 \begin{itemize}[nosep,after=\strut]
%                     \item ID
%                     \item Название
%                     \item Популярность
%                     \item Стоимость
%                 \end{itemize}
%             \end{minipage}
%         \\ \hline

%         Хостинг &
%             \begin{minipage}[t]{\linewidth}
%                 \begin{itemize}[nosep,after=\strut]
%                     \item ID
%                     \item Название
%                     \item Плата в месяц
%                 \end{itemize}
%             \end{minipage}
%         \\ \hline

%         Игрок &
%             \begin{minipage}[t]{\linewidth}
%                 \begin{itemize}[nosep,after=\strut]
%                     \item ID
%                     \item Никнейм
%                     \item Сыграно часов на сервере
%                     \item Дата последнего захода на сервер
%                 \end{itemize}
%             \end{minipage}
%         \\ \hline

%         Пользователь &
%             \begin{minipage}[t]{\linewidth}
%                 \begin{itemize}[nosep,after=\strut]
%                     \item ID
%                     \item Логин
%                     \item Пароль
%                     \item Роль
%                 \end{itemize}
%             \end{minipage}
%         \\ \hline
% \end{longtable}
% \end{center}

Также на рисунке \ref{img:er} предствалена ER-диаграмма \cite{er} разрабатываемой системы в нотации Чена.

\imgs{er}{h!}{0.3}{ER-диаграмма в нотации Чена}


\section{Выбор модели базы данных}

База данных \cite{db-is} --- упорядоченный набор структурированныой информации или данных, которые обычно хранятся в электронном виде в компьютерной системе. База данных обычно управляется системой управления базами данных (СУБД). Данные вместе с СУБД, а также приложения, которые с ними связаны, называются базой данных.

Данные в наиболее распространенных типах современных баз данных обычно хранятся в виде строк и столбцов формирующих таблицу \cite{db-is}. Этими данными можно управлять, изменять, обновлять, контролировать и упорядочивать.

Базы данных делятся на три модели организации данных:

\begin{itemize}
    \item дореляционные;
    \item реляционные;
    \item постреляционные.
\end{itemize}


\subsection{Дореляционные модели}

Дореляционные модели базы данных \cite{pre_sql} предоставляли доступ на уровне записей, которые располагались в виде древовидной структуры со связями предок-потомок. При этом взаимодействие с базой данных происходило с использованием языков программирования, которые были расширены функциями дореляционных СУБД. 

Главными недостатками является то, что оптимизация доступа к данным со стороны системы отсутствует, а также древовидная структура является весьма трудоемкой.


\subsection{Реляционные модели}

Реляционная база данных \cite{db-sql} --- совокупность отношений, которые содержат всю информацию, которая должна храниться в базе данных. Каждое отношение --- двумерная таблица, в каждой строке которой хранится запись об объекте, а в каждом столбце --- свойства данного объекта.

Реляционные базы данных обладают несомненным преимуществом -- благодаря стандартизированного языка запросов SQL, существует возможность подмены СУБД.


\subsection{Постреляционные модели}

В постреляционных моделях баз данных \cite{db-nosql} не используется табличная схема строк и столбцов. В этих базах данных применяется модель хранения, оптимизированная под конкретные требования типа хранимых данных.

При этом они делятся на следующие основные категории:

\begin{itemize}
    \item коллекции --- документы, упорядоченные по группам;
    \item ключ-значение --- хэш-таблица, в которой по ключу находится значение;
    \item колоночная --- хранит информацию в виде разреженной матрицы, строки и столбцы которой используются как ключи;
    \item графовые --- сетевая база, использующие узлы и ребра для хранения данных.
\end{itemize}

Данные модели используются для спефифических задач, где явно подходит одна из приведенных выше категорий, что явно ускоряет работу программного продукта благодаря грамотной работе с данными.


\subsection*{Вывод}

\textbf{В РАЗРАБОТКЕ:} СОЗДАТЬ ТАБЛИЦУ (надо ли?)

Для поставленной задачи разработки информационной системы наилучшим обрзаом подходит реляционная модель хранения данных. Выбор обусловлен необходимостью хранить данные в виде структированных таблиц, между которыми проведена связь. 





% \subsection{Реляционные базы данных}

% Реляционная база данных \cite{db-sql} --- совокупность отношений, которые содержат всю информацию, которая должна храниться в базе данных. Каждое отношение --- двумерная таблица, в каждой строке которой хранится запись об объекте, а в каждом столбце --- свойства данного объекта. При этом для каждой записи обязательно должен присутсвовать атрибут --- первичный ключ. Он однозначно идентифицирует запись в таблице базы данных.

% В реляционных базах данных соблюдается принцип ACID:

% \begin{itemize}
%     \item атомарность --- для каждой транзакции либо выполняются все опреации внутри нее, либо ни одной;
%     \item согласованность --- транзакция не может перевести базу данных в несогласованное состояние;
%     \item изолированность --- на результат транзакции не влияюи другие транзакции;
%     \item долговечность --- любые изменеия, которые были внесены транзакцией, остаются в базе данных при любой ситуации.
% \end{itemize}

% Реляционные базы данных обладают рядом преимуществ:

% \begin{itemize}
%     \item использование стандартизированного языка запросов SQL \cite{sql-is};
%     \item отображает всю информацию в простой и единой форме --- отношении;
%     \item основано на математической модели, которая строго описывает основные операции над даннымиl.
% \end{itemize}

% Недостатками реляционные баз данных являются:

% \begin{itemize}
%     \item медленный доступ к данным;
%     \item вертикальная масштабируемость;
%     \item трудоемкость разработки.
% \end{itemize}


% \subsection{NoSQL базы данных}

% NoSQL база данных \cite{db-nosql} --- база данных, в которой в отличие от большинства традиционных систем баз данных не используется табличная схема строк и столбцов. В этих базах данных применяется модель хранения, оптимизированная под конкретные требования типа хранимых данных.

% NoSQL базы данных делятся на 4 типа:

% \begin{itemize}
%     \item коллекции --- документы, упорядоченные по группам;
%     \item ключ-значение --- хэш-таблица, в которой по ключу находится значение;
%     \item колоночная --- хранит информацию в виде разреженной матрицы, строки и столбцы которой используются как ключи;
%     \item графовые --- сетевая база, использующие узлы и ребра для хранения данных.
% \end{itemize}

% NoSQL базы данных обладают следующими преимуществами:

% \begin{itemize}
%     \item высокая скорость доступа к данным;
%     \item горизонтальная масштабируемость;
%     \item хранить информацию можно неструктурированной.
% \end{itemize}

% При этом NoSQL базы данных обладают рядом недостатков:

% \begin{itemize}
%     \item привязка к одной СУБД, так как нет стандартизированного языка запросов;
%     \item смягчение требований ACID.
% \end{itemize}


\section{Вывод}

В данном разделе была проанализирована выполняемая задача --- была проведена ее формализация, проведена формализация данных, описаны типы пользователей. Также были расммотрены модели базы данных и выбрана реляционная модель.
