\chapter*{Заключение}
\addcontentsline{toc}{chapter}{Заключение}

В ходе программного обеспечения было разработано программное обеспечение, которое реализует модель водопада по методу частиц. Полученное приложение позволяет изменять параметры водопада в интерактивном режиме (скорость течения воды, угол наклона падающей воды, высота водопада, размер частиц, количество частиц). В процессе выполнения данной работы были выполнены следующие задачи:

\begin{itemize}
    \item рассмотрены методы реализации модели водопада;
    \item выбран алгоритм, который наиболее эффективно решает поставленную задачу;
    \item реализован выбранный алгоритм;
    \item разработана структура классов проекта;
    \item проведен эксперимент по замеру производительности полученного программного обеспечения.
\end{itemize}

В процессе исследования было выявлено, что производительность программы понижается экспоненциально при линейном увеличении количества частиц. При этом водопад визуально выглядит лучше при наибольшем количестве частиц, которые вместе составляют единую структуру.
