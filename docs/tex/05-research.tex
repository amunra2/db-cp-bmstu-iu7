\chapter{Исследовательская часть}

\section{Цель эксперимента}

Целью эксперимента является сравнение времени выполнения запросов при использовании индексов и без них. 

Созданные индексы будут иметь тип \textit{BTree}. Данный тип индексов используется в ситуациях, когда данные можно отсортировать.

Эксперименты проводились на таблицах, которые имеют более 300 записей, чтобы продемонстрировать применение индексов с наибольшей наглядностью.

\section{Технические характеристики}

Технические характеристики устройства, на котором выполнялось тестирование представлены далее:

\begin{itemize}
    \item операционная система: Ubuntu 22.04.1 \cite{ubuntu} Linux \cite{linux} x86\_64;
    \item память: 8 GiB;
    \item процессор: Intel Core i5-7300HQ CPU @ 2.50GHz \cite{intel};
    \item видеокарта: NVIDIA GeForce GTX 1050Ti with 4 GB GDDR5 Dedicated VRAM \cite{gtx1050}.
\end{itemize}

При тестировании ноутбук был включен в сеть питания. При этом ноутбук был нагружен только встроенными приложениями и системой тестирования. 


\section{Результаты эксперимента}

\subsection{Поиск по первичному ключу}

На рисунке \ref{img:pk_noindex} представлены замеры времени выполнения запроса при поиске записи в таблице <<Server>> по первичному ключу.
\imgs{pk_noindex}{h!}{0.45}{Поиск по первичному ключу (без индекса)}

В листинге \ref{lst:pk_index} показано создание индекса <<server\_id\_index>> для первичного ключа таблицы <<Server>>. 
\mylisting[sql]{indexes.sql}
            {firstline=2,lastline=2}{Индекс для первичного ключа таблицы <<Server>>}{pk_index}{}

При этом на рисунке \ref{img:pk_index} представлены результаты замера времени запроса поиска записи в таблице <<Server>> после создания индекса.
\imgs{pk_index}{h!}{0.4}{Поиск по первичному ключу (c индексом)}

Как видно из результата, создание индекса для первичного ключа не принесло прироста производительности выполнения запроса. Это связано с тем, что подобные индексы ао умолчанию создаются для первичных ключей таблиц в базе данных.


\subsection{Фильтрация по столбцу таблицы}

На рисунке \ref{img:rating_noindex} приведены замеры времени выполнения запроса при фильтрации записей в таблице <<Server>> по столбцу <<Rating>>.
\imgs{rating_noindex}{h!}{0.45}{Фильтрация по столбцу <<Rating>> (без индекса)}

В листинге \ref{lst:rating_index} показано создание индекса <<server\_rating\_index>> для столбца <<Rating>> таблицы <<Server>>. 
\mylisting[sql]{indexes.sql}
            {firstline=5,lastline=5}{Индекс для столбца <<Ratung>> таблицы <<Server>>}{rating_index}{}

После создания индекса, были снова выполнены замеры времени выполнения запроса с фильтрацей записей в таблице <<Server>> по пстолбцу <<Rating>>, что показано на рисунке \ref{img:rating_index}.
\imgs{rating_index}{h!}{0.4}{Фильтрация по столбцу <<Rating>> (c индексом)}

Из полученного результита видно, что время выполнения запроса уменьшилось примерно в 4.19 раз после создания индекса для столбца <<Rating>>.


\section*{Вывод}

Как видно из результатов, применение индексов позволяет ускорить выполнение запросов. При этом индексы не следует создавать для первичных ключей таблиц, так как индексы для них создаются по умолчанию. Таким образом, индексы являются важным инструментов оптимизации работы с базой данных.
